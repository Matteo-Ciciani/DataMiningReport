%%%%%%%%%%%%%%%%%%%%%%%%%%%%%%%%%%%%%%%%%
% Stylish Article
% LaTeX Template
% Version 2.1 (1/10/15)
%
% This template has been downloaded from:
% http://www.LaTeXTemplates.com
%
% Original author:
% Mathias Legrand (legrand.mathias@gmail.com) 
% With extensive modifications by:
% Vel (vel@latextemplates.com)
%
% License:
% CC BY-NC-SA 3.0 (http://creativecommons.org/licenses/by-nc-sa/3.0/)
%
%%%%%%%%%%%%%%%%%%%%%%%%%%%%%%%%%%%%%%%%%

%----------------------------------------------------------------------------------------
%	PACKAGES AND OTHER DOCUMENT CONFIGURATIONS
%----------------------------------------------------------------------------------------
\PassOptionsToPackage{table}{xcolor}
\PassOptionsToPackage{dvipsnames}{xcolor}

\documentclass[fleqn,10pt]{SelfArx} % Document font size and equations flushed left

\usepackage[english]{babel} % Specify a different language here - english by default

\usepackage{lipsum} % Required to insert dummy text. To be removed otherwise
\usepackage{tabularx}
\usepackage{multirow}
\setlength{\parindent}{0pt}

%----------------------------------------------------------------------------------------
%	COLUMNS
%----------------------------------------------------------------------------------------

\setlength{\columnsep}{0.55cm} % Distance between the two columns of text
\setlength{\fboxrule}{0.75pt} % Width of the border around the abstract

%----------------------------------------------------------------------------------------
%	COLORS
%----------------------------------------------------------------------------------------

\definecolor{color1}{RGB}{0,0,100} % Color of the article title and sections
\definecolor{color2}{RGB}{0,10,10} % Color of the boxes behind the abstract and headings

%----------------------------------------------------------------------------------------
%	HYPERLINKS
%----------------------------------------------------------------------------------------

\usepackage{hyperref} % Required for hyperlinks
\hypersetup{hidelinks,colorlinks,breaklinks=true,urlcolor=color2,citecolor=color1,linkcolor=color1,bookmarksopen=false,pdftitle={Title},pdfauthor={Author}}

%----------------------------------------------------------------------------------------
%	ARTICLE INFORMATION
%----------------------------------------------------------------------------------------

\JournalInfo{Laboratory of Data Mining Report} % Journal information
\Archive{} % Additional notes (e.g. copyright, DOI, review/research article)

\PaperTitle{Identification of putative drug targets for coronary artery disease} % Article title

\Authors{John Smith\textsuperscript{1}*, James Smith\textsuperscript{2}} % Authors
\affiliation{\textsuperscript{1}\textit{Department of Biology, University of Examples, London, United Kingdom}} % Author affiliation
\affiliation{\textsuperscript{2}\textit{Department of Chemistry, University of Examples, London, United Kingdom}} % Author affiliation
\affiliation{*\textbf{Corresponding author}: john@smith.com} % Corresponding author

\Keywords{Keyword1 --- Keyword2 --- Keyword3} % Keywords - if you don't want any simply remove all the text between the curly brackets
\newcommand{\keywordname}{Keywords} % Defines the keywords heading name

%----------------------------------------------------------------------------------------
%	ABSTRACT
%----------------------------------------------------------------------------------------

\Abstract{\lipsum[1]~}

%----------------------------------------------------------------------------------------

\begin{document}

\flushbottom % Makes all text pages the same height

\maketitle % Print the title and abstract box

\tableofcontents % Print the contents section

\thispagestyle{empty} % Removes page numbering from the first page

%----------------------------------------------------------------------------------------
%	ARTICLE CONTENTS
%----------------------------------------------------------------------------------------

\section*{Introduction} % The \section*{} command stops section numbering

\addcontentsline{toc}{section}{Introduction} % Adds this section to the table of contents

Drug repurposing is a strategy to identify new therapeutics uses for marketed drugs \cite{Polamreddy}. This procedure has known for time but only in recent years its use has increased due to new molecular discoveries, new high-throughput technologies and databases. In average the launch of a new drug on the market takes 14 years, drug repurposing can drastically decrease this time because of the usage of previous knowledge and, moreover, it can lower the costs of the R{\&}D process.\\
\\
Coronary artery disease (CAD) is the most common type of heart disease.It is the leading cause of death in the United States and in Italy (WHO data). The condition leads the formation of a waxy substance calle plaque into coronary arteries decreasing the flow of oxygen-rich blood to the heart. The condition could worsen after the plaque rupture due to the formation of blood clot that could mostrly (or completely) block blood flow through a coronary artery. The disease can weaken the heart muscle and lead to heart failure and arrhytmias, moreover it cas cause angina and heart attack.\\
\\
In this project we apply NES2RA\cite{NES2RA}, a pipeline that finds candidate genes to expand a known gene-network based PC-algorithm \cite{PC-alg}, to find marketed drug that can be repurposed for CAD. 
The pipeline of our work can be divided in 2 significant steps: retrieving known target genes, backward expansion and analysis.\\
\\
In the first step the ranked CAD genetic associated genes are obtained using Open Target \cite{OpenTargetPlatform}, we decide to focus on the 100 most important ones without minding if there are already known drug target for the disease.\\
For each gene the corresponding isoforms are picked from the filtered annotation of the FANTOM5 database \cite{fantom} and then expanded by NES2RA using the Boinc Platform \cite{realBoinc} and FANTOM5. NES2RA gives an expansion file for each isoform, where isoforms related with the one in input are collected and ranked by an association score. Minding the importance of the previous 100 CAD genes and the overall score for each isoform , the expansion files are filtered such that a list of particularly interesting isoforms is obtained. Successively using the Open Target Platform, we extract the gens that are already known targets for any drug from the aforementioned list. At the end of this step we are left with a list of 148 drugged genes corresponding to 217 isoforms.\\
\\
The purpose of the second step is to verify which of the genes has a drug that can be repurposed for CAD.
Firstly a backward expansion on the 217 isoforms is performed using NES2RA.
Secondly an analsyis using weighted Jaccard Similarity (WGS) \cite{WGS} for each expansion result paired with the initial CAD genetic associated genes is done to obtain a p-value for possible relation among the isoform in input and the CAD. 

The aim of the project is to see if NES2RA can be proposed as a new \textit{in silico} tool for drug repurposing and to find possible new drugs for the CAD disease.  


\section*{Materials and Methods}
\addcontentsline{toc}{section}{Materials and Methods} % Adds this section to the table of contents

\subsection*{Platforms and Datasources}
\addcontentsline{toc}{subsection}{Platforms and Datasources} % Adds this section to the table of contents

Nessra is designed in three steps in which first subsets data,
secondly runs the skeleton of the PC algorithm and finally aggregates results, computing the frequencies of the transcripts.

For the most computationally demanding step of Nessra, which consists in running the skeleton of the PC algorithm, the sigle gene expansions were run within the local gene@home \cite{boinc} BOINC (Berkeley Open Infrastructure for Network Computing) project, hosted by the TN-Grid platform. 
The platform allowed to distribute the expansions computational burden to committed volunteers' machines in order to reduce the running time and memory usage of the algorithm as much as possible.
This gene@home project was previously set up for the network expansion of genes whose expression data came from the FANTOM5 Database \cite{fantom}, so this guided the choice of the transcriptional data to run Nessra on.
To adress the drug repurposing task, Open Targets\cite{open} platform was chosen to select genes involved in the CAD disease.  


\subsection*{Pipeline}
\addcontentsline{toc}{subsection}{Pipeline} % Adds this section to the table of contents


general description of the pipeline and figure

\begin{figure}
	\includegraphics[width=1.1\linewidth, trim = 2cm 0 1.5cm 0.4cm, clip = true]{pipeline.png}
	\caption{Pipeline sketch. Color legend = ...}
	\label{Fig:pipe}
\end{figure}

Enri

selection of genes to expand

nessra expansion of cad associated genes

score aggregation of each isoform

filter on isoforms

Matteo

The resulting n isoforms were ranked according to the aggregated score described previously. Then, the ranked list of isoforms was converted in a ranked list of genes. The rank of a gene was obtained as the minimum of the ranks of its isoforms.

\subsubsection{Selection of target genes}

The Open Tragets pyhton API [reference?] was used to query which of these genes are known to be targets of clinically approved drugs. We obtained a list of 148 target genes, which are the only ones that are considered in the following analyses.

\subsubsection{NES$^{\textbf{2}}$RA expansion of target genes}

In order to identify a subset of genes with a strong association with CAD, we selected the isoforms corresponding to these 148 target genes and we performed single isoform expansions using NES$^2$RA. Only the isoforms that passed the previous filter were considered. We expect NES$^2$RA expansions of genes that have strong interaction with CAD genes to have a large overlap with the list of CAD genes we started form.

\subsubsection{Comparison of expanded lists to CAD genes}

In order to quantify the overlap between the ranked list of CAD genes and the ranked expansion lists previously obtained, we used the weighted Jaccard similarity (WJS), which we define below.\medskip

\noindent
\textit{Definition: given two weighted list of items, $\rho$ and $\sigma$, of equal length $N$, their weighted Jaccard similarity, $WJS(\rho, \sigma)$, is defined as:}

$$
WJS(\rho, \sigma) = \dfrac{\sum_{i=1}^Nmin(\rho_i,\sigma_i)}{\sum_{i=1}^Nmax(\rho_i,\sigma_i)}
$$

\noindent
\textit{where $\rho_i$ and $\sigma_i$ are the weights corresponding to the same feature $i$.}\medskip

In our analysis the weight of a feature $i$ (gene or isoform) in a ranked list $\rho$ is computed as $length(\rho) - rank_{\rho}(i) + 1$. This allows to assign large weights to high ranking features.

Since in principle the lists of features that we want to compare with the WJS do not contain the same elements, we need to add the missing elements to those lists. Given two lists of features $\rho$ and $\sigma$, containing different elements, we added the missing elements of one list to the other and viceversa. The missing elements are added as ties in the last position of the ranking.

In order to compute the WJS of the ranked list of CAD genes and a ranked list of isoforms obtained using NES$^2$RA, we have to either convert the ranked list of CAD genes in a ranked list of isoforms or convert the ranked list of isoforms obtained using NES$^2$RA in a ranked list of genes. We explored both options:

\begin{itemize}
	\item \textit{Convert NES$^2$RA isoforms into genes}: Given list of isoforms obtained with NES$^2$RA, we can convert it in a list of genes, assigning to each gene the largest relative frequency of its isoforms. Then the genes were ranked according to relative frequency.
	\item \textit{Convert CAD genes into isoforms}: The ranked list of CAD genes was converted to a weighted list of isoforms, were the weight of each isoform corresponded to the rank of the genes. Then the list of isoforms was ranked according to the weights. This way, isoforms corresponding to the same gene presented the same rank.
\end{itemize}

We computed the WJS of the ranked list of CAD genes (respectively, isoforms) and the ranked lists of genes (respectively, isoforms) obtained with NES$^2$RA.

\subsubsection{Significance analysis} 

The scores obtained in different comparisons are not directly comparable between them, since they depend on the length of the lists. In order to obtain values that can be compared directly, we used a permutation approach to estimate the distribution of scores of lists of different length. For each length present, we generated 2000 random lists of genes or isoforms, we computed the WJS and we generated a distribution of the scores. Then, we used this distributions to compute p-values.

Using a significance level of 0.05 (?), we obtained n genes and m isoforms with expansion lists enriched in CAD gene. The list of isoforms was also converted in a list of genes.

\subsubsection{Enrichment analysis}

In order to gain insight on the biological role of the identified genes, we performed a functional enrichment analysis...


\section*{Results}
\addcontentsline{toc}{section}{Results} % Adds this section to the table of contents
Matteo

The list of 149 target genes is reported in the Supplementary Material (Tab. \ref{tab:194genes}).

The list of 36 significant genes are reported in Tab. \ref{tab:sign}. In this table are also indicated the genes found by option2.

\renewcommand{\arraystretch}{1.1}

\newcolumntype{R}{>{\raggedleft\arraybackslash}X}
\begin{table}[htb]
	\centering
	\rowcolors{2}{NavyBlue!10}{white}
	\begin{tabularx}{\linewidth}{XRRRR}
		\rowcolor{NavyBlue!80}
		\textbf{\color{white} Gene} & \textbf{\color{white} Number of isoforms} & \textbf{\color{white} Minimum p-value} & \textbf{\color{white} Number of isoforms} & \textbf{\color{white} Minimum p-value} \\ 
		CASP7 & 1 & 0 & 1 & 0 \\ 
		COL1A1 & 1 & 0 & 1 & 0 \\ 
		COL4A1 & 4 & 0 & 4 & 0 \\ 
		COL4A2 & 6 & 0 & 3 & 0 \\ 
		CTGF & 2 & 0 & 1 & 0 \\ 
		FN1 & 8 & 0 & 8 & 0 \\ 
		IL1R1 & 1 & 0 & 1 & 0 \\ 
		INSR & 1 & 0 & 1 & 4.0e-02 \\ 
		JAK1 & 1 & 0 &  &  \\ 
		NDUFA4L2 & 1 & 0 & 1 & 0 \\ 
		PDE4B & 2 & 0 & 1 & 1.7e-02 \\ 
		PDE4D & 3 & 0 & 2 & 7.5e-03 \\ 
		RARA & 1 & 0 &  &  \\ 
		TLR4 & 2 & 0 &  &  \\ 
		PIK3CG & 1 & 1.2e-02 &  &  \\ 
		FNTB & 1 & 1.6e-02 &  &  \\ 
		CSF1 & 1 & 1.7e-02 &  &  \\ 
		TSG101 & 1 & 1.7e-02 &  &  \\ 
		F5 & 1 & 2.3e-02 &  &  \\ 
		CDK14 & 1 & 2.6e-02 & 1 & 2.2e-02 \\ 
		TGFB1 & 2 & 2.7e-02 &  &  \\ 
		ANPEP & 1 & 2.9e-03 &  &  \\ 
		AVPR1A & 1 & 2.9e-03 & 1 & 4.1e-03 \\ 
		DNMT3A & 1 & 2.9e-03 &  &  \\ 
		PIK3CA & 1 & 2.9e-03 & 1 & 3.9e-02 \\ 
		TNFRSF1A & 1 & 2.9e-03 &  &  \\ 
		IL6 & 1 & 3.2e-02 & 1 & 1.7e-02 \\ 
		FLT1 & 1 & 3.5e-02 &  &  \\ 
		SEBOX & 1 & 3.8e-02 &  &  \\ 
		COL5A3 & 1 & 4.4e-02 &  &  \\ 
		GAA & 1 & 4.4e-02 &  &  \\ 
		KCNJ2 & 1 & 4.6e-02 & 1 & 1.1e-02 \\ 
		FGA & 1 & 4.9e-02 &  &  \\ 
		NAMPT & 1 & 5.3e-03 &  &  \\ 
		IFNAR2 & 1 & 7.5e-03 &  &  \\ 
		PTGER4 & 1 & 7.5e-03 &  &  \\ 
	\end{tabularx}
	\smallskip
	\caption{resulting genes}
	\label{tab:sign}
\end{table}

The results of the enrichment analysis against the Biological Processes form Gene Ontology are reported in Tab. \ref{tab:GO36_1} and Tab. \ref{tab:GO36_2} for the results of option1 and in Tab. \ref*{tab:GO18} for the results of option2.

The drugs that target the genes resulting from option1 are reported in Tab. \ref{tab:drugs}.

19 genes found were already associated with CAD. 4 of them are also target of drugs used to treat CAD. The list of genes and their association levels are reported in Fig. \ref{fig:19CAD}.


\begin{figure}[ht]
	\includegraphics[width=\linewidth]{already_in}
	\smallskip
	\caption{Putative targets already associated with CAD. Four of them are also already targeted in CAD.}
	\label{fig:19CAD}
\end{figure}

results of the enrichment analysis

\section*{Discussion}
\addcontentsline{toc}{section}{Discussion} % Adds this section to the table of contents
Chiara

gene that are already targets of drugs used for cad

interpretation of the enrichment analysis

discussion of new targets and drugs

\section*{Conclusions}
\addcontentsline{toc}{section}{Conclusions} % Adds this section to the table of contents

necessity of experimental validations

\phantomsection
%----------------------------------------------------------------------------------------
%	REFERENCE LIST
%----------------------------------------------------------------------------------------
\phantomsection
\bibliographystyle{unsrt}
%\bibliography{biblio}
\begin{thebibliography}{1}
	
	\bibitem{Polamreddy} Polamreddy P, Gattu N. The drug repurposing landscape from 2012 to 2017: evolution, challenges, and possible solutions. Drug Discov Today. 2018 Dec 1.	

	\bibitem{NES2RA} Asnicar F, Masera L, NES2RA: network expansion by stratified variable subsetting and ranking aggregation. INTERNATIONAL JOURNAL OF HIGH PERFORMANCE COMPUTING APPLICATIONS (2018)
	
	\bibitem{PC-alg}Spirtes P and Glymour C An algorithm for fast recovery
of sparse causal graphs. Social Science Computer, Review, (1991).  

	\bibitem{OpenTargetPlatform}Denise Carvalho-Silva, Open Targets Platform: new developments and updates two years on, Nucleic Acids Research 

	\bibitem{realBoinc}Anderson DP, BOINC: A system for public-resource computing and storage. In: Proceedings of the 5th IEEE/ ACM international workshop on grid computing, GRID ‘04,Washington, DC, US (2004)
	
	\bibitem{WGS}Improved Consistent Sampling, Weighted Minhash and L1 Sketching, Sergey Ioffe, 2010, ICDM
	
	\bibitem{boinc} Asnicar, F. et al. NES2RA. Int. J. High Perform. Comput. Appl. 32, 380–392 (2016).
	
	
	\bibitem{fantom} Noguchi, S. et al. FANTOM5 CAGE profiles of human and mouse samples. Sci Data 4, 170112 (2017).
	
	
	\bibitem{open} Koscielny, G. et al. Open Targets: a platform for therapeutic target identification and validation. Nucleic Acids Res. 45, D985–D994 (2017).
	
	
\end{thebibliography}

%----------------------------------------------------------------------------------------
%	SUPPLEMENTARY MATERIAL
%----------------------------------------------------------------------------------------

\pagebreak
\onecolumn

\section*{SUPPLEMENTARY MATERIAL}

\addcontentsline{toc}{section}{Supplementary Material} % Adds this section to the table of contents


\begin{table}[ht]
	\centering
	\rowcolors{2}{NavyBlue!10}{white}
	\begin{tabularx}{\textwidth}{XXXXXX}
		NDUFA4L2 & JAK1 & CDK16 & SRD5A1 & PRKCD & FLT4 \\ 
		FN1 & KCNJ2 & PIM2 & FGFR4 & ABCA1 & DHODH \\ 
		COL4A1 & FDPS & COL27A1 & TUBA1B & ERBB3 & FNTB \\ 
		COL4A2 & CPT2 & ALOX5 & PIK3CG & NR3C1 & TNFSF12 \\ 
		CTGF & LAMB3 & CD70 & GABBR1 & HGF & TGFB3 \\ 
		F5 & PCSK9 & GSK3B & PDE2A & BIRC2 & KCNH8 \\ 
		SEBOX & FLT1 & SLC29A1 & MMP9 & ERAP1 & EGLN2 \\ 
		TGFB1 & VEGFB & PIM3 & PSCA & OPRL1 & METAP2 \\ 
		F2 & POLD1 & CSF1 & CALM1 & GSTP1 & LAMA2 \\ 
		TLR4 & IL1R1 & CASP7 & PDE8B & PRKACA & NDUFA13 \\ 
		COL5A3 & XPNPEP2 & GAA & DNMT3A & RAMP2 & MAPKAPK5 \\ 
		COL1A1 & NR1H3 & PDE4B & IL6 & HSP90AA1 & CD276 \\ 
		FGA & NFE2L2 & PTGS2 & EPHA2 & INSR & HDAC5 \\ 
		FOLR1 & GBA & NCSTN & PIK3CA & MAP3K9 & DRD2 \\ 
		DHCR24 & PLA2G2A & NDUFS2 & TNFRSF8 & TNFSF13 & NTRK1 \\ 
		GLP2R & ACAT1 & TSG101 & LTB4R & CASP8 & PIK3R3 \\ 
		TNFRSF1A & FDFT1 & PLK1 & VDR & CACNG4 & MALT1 \\ 
		AVPR1A & HLA-DRB1 & NDUFB7 & EEF2 & NFE2 & SLC6A9 \\ 
		FGG & IFNGR2 & BCL2L1 & SCN9A & LTBR & HDAC6 \\ 
		NAMPTL & PLA2G7 & IDH2 & PTH1R & EPHB6 & PTGER4 \\ 
		SRD5A3 & CHEK2 & PSMA4 & P4HB & GART & KCNK10 \\ 
		PDE4D & ATP1B3 & ENG & CFD & KCNH2 & CD44 \\ 
		HAMP & ANPEP & EPHA1 & ARAF & S1PR5 & P4HTM \\ 
		RARA & COL6A3 & TNFRSF10B & SIGMAR1 & TUBB & MET \\ 
		NAMPT & COL11A2 & CDK14 & IFNAR2 & KCNC3 &  \\ 
	\end{tabularx}
	\smallskip
	\caption{List of tagret genes}
	\label{tab:194genes}
\end{table}

\renewcommand{\arraystretch}{1.2}

\begin{table}[ht]
	\centering
	\scriptsize
	\rowcolors{2}{white}{NavyBlue!10}
	\begin{tabularx}{\textwidth}{lXlll}
		\rowcolor{NavyBlue!80}
		 \textbf{\color{white} DRUG} & \textbf{\color{white} TARGET} & \textbf{\color{white} DISEASE} & \textbf{\color{white} PAHSE} & \textbf{\color{white} MOLECULE TYPE} \\
		SORAFENIB & FLT1 & neoplasm & Phase IV & Small molecule \\ 
		COLLAGENASE CLOSTRIDIUM HISTOLYTICUM & COL5A3 & Dupuytren Contracture & Phase IV & Enzyme \\ 
		APREMILAST & PDE4D & immune system disease & Phase IV & Small molecule \\ 
		INSULIN GLARGINE & INSR & type II diabetes mellitus & Phase IV & Protein \\ 
		METFORMIN & NDUFA4L2 & type II diabetes mellitus & Phase IV & Small molecule \\ 
		INSULIN HUMAN & INSR & type II diabetes mellitus & Phase IV & Protein \\ 
		ADAPALENE & RARA & acne & Phase IV & Small molecule \\ 
		THEOPHYLLINE & PDE4B & Bronchiectasis & Phase IV & Small molecule \\ 
		LENVATINIB & FLT1 & thyroid carcinoma & Phase IV & Small molecule \\ 
		INSULIN PORK & INSR & diabetes mellitus & Phase IV & Protein \\ 
		PEGINTERFERON ALFA-2A & IFNAR2 & hepatitis C infection & Phase IV & Protein \\ 
		INTERFERON BETA-1A & IFNAR2 & multiple sclerosis & Phase IV & Protein \\ 
		FLAVOXATE & PDE4D & pain & Phase IV & Small molecule \\ 
		DECITABINE & DNMT3A & chronic myelomonocytic leukemia & Phase IV & Small molecule \\ 
		AZACITIDINE & DNMT3A & myelodysplastic syndrome & Phase IV & Small molecule \\ 
		INSULIN LISPRO & INSR & type I diabetes mellitus & Phase IV & Protein \\ 
		ACITRETIN & RARA & squamous cell carcinoma & Phase IV & Small molecule \\ 
		TRETINOIN & RARA & acne & Phase IV & Small molecule \\ 
		PENTOXIFYLLINE & PDE4B & chronic kidney disease & Phase IV & Small molecule \\ 
		PEGINTERFERON ALFA-2B & IFNAR2 & Chronic Hepatitis C infection & Phase IV & Protein \\ 
		INSULIN DETEMIR & INSR & diabetes mellitus & Phase IV & Protein \\ 
		DRONEDARONE & KCNJ2 & atrial fibrillation & Phase IV & Small molecule \\ 
		ISOTRETINOIN & RARA & seborrheic dermatitis & Phase IV & Small molecule \\ 
		DIPYRIDAMOLE & PDE4D & coronary artery disease & Phase IV & Small molecule \\ 
		INSULIN ASPART & INSR & type II diabetes mellitus & Phase IV & Protein \\ 
		CRISABOROLE & PDE4B & seborrheic dermatitis & Phase IV & Small molecule \\ 
		SUNITINIB & FLT1 & neoplasm & Phase IV & Small molecule \\ 
		INSULIN GLULISINE & INSR & type I diabetes mellitus & Phase IV & Protein \\ 
		INSULIN SUSP ISOPHANE BEEF & INSR & type I diabetes mellitus & Phase IV & Protein \\ 
		NINTEDANIB & FLT1 & idiopathic pulmonary fibrosis & Phase IV & Small molecule \\ 
		ROFLUMILAST & PDE4B & chronic obstructive pulmonary disease & Phase IV & Small molecule \\ 
		INTERFERON BETA-1B & IFNAR2 & relapsing-remitting multiple sclerosis & Phase IV & Protein \\ 
		CONIVAPTAN & AVPR1A & heart failure & Phase IV & Small molecule \\ 
		TAZAROTENE & RARA & acne & Phase IV & Small molecule \\ 
		TOFACITINIB & JAK1 & rheumatoid arthritis & Phase IV & Small molecule \\ 
		ETRETINATE & RARA & psoriasis & Phase IV & Small molecule \\ 
		INSULIN DEGLUDEC & INSR & type II diabetes mellitus & Phase IV & Protein \\ 
		PAZOPANIB & FLT1 & renal cell carcinoma & Phase IV & Small molecule \\ 
		INSULIN ASPART PROTAMINE RECOMBINANT & INSR & type II diabetes mellitus & Phase IV & Protein \\ 
		SILTUXIMAB & IL6 & Giant Lymph Node Hyperplasia & Phase IV & Antibody \\ 
		PEGINTERFERON BETA-1A & IFNAR2 & multiple sclerosis & Phase IV & Protein \\ 
		INSULIN PURIFIED PORK & INSR & diabetes mellitus & Phase IV & Protein \\ 
		ANAKINRA & IL1R1 & immune system disease & Phase IV & Protein \\ 
		INTERFERON ALFA-2B & IFNAR2 & Chronic Hepatitis C infection & Phase IV & Protein \\ 
		FIBRINOLYSIN, HUMAN & FGA & Recurrent thrombophlebitis & Phase IV & Unknown \\ 
		VASOPRESSIN & AVPR1A & sudden cardiac arrest & Phase IV & Protein \\ 
		DROTRECOGIN ALFA (ACTIVATED) & F5 & sepsis & Phase IV & Unknown \\ 
		INTERFERON ALFACON-1 & IFNAR2 & hepatitis C infection & Phase IV & Protein \\ 
		INSULIN SUSP ISOPHANE RECOMBINANT HUMAN & INSR & diabetic ketoacidosis & Phase IV & Protein \\ 
		MIGLITOL & GAA & diabetes mellitus & Phase IV & Small molecule \\ 
		INSULIN SUSP ISOPHANE SEMISYNTHETIC PURIFIED HUMAN & INSR & diabetes mellitus & Phase IV & Protein \\ 
		TIVOZANIB & FLT1 & neoplasm & Phase IV & Small molecule \\ 
		AXITINIB & FLT1 & neoplasm & Phase IV & Small molecule \\ 
		REGORAFENIB & FLT1 & neoplasm & Phase IV & Small molecule \\ 
		DESMOPRESSIN & AVPR1A & hemorrhage & Phase IV & Protein \\ 
		RUXOLITINIB & JAK1 & polycythemia vera & Phase IV & Small molecule \\ 
		INSULIN PURIFIED BEEF & INSR & diabetes mellitus & Phase IV & Protein \\ 
		VANDETANIB & FLT1 & thyroid carcinoma & Phase IV & Small molecule \\ 
		ALITRETINOIN & RARA & neoplasm & Phase IV & Small molecule \\ 
		INSULIN LISPRO PROTAMINE RECOMBINANT & INSR & type II diabetes mellitus & Phase IV & Protein \\ 
		DYPHYLLINE & PDE4B & obstructive lung disease & Phase IV & Small molecule \\ 
		BARICITINIB & JAK1 & immune system disease & Phase IV & Small molecule \\ 
		AMLEXANOX & PDE4D & obstructive lung disease & Phase IV & Small molecule \\ 
		OCRIPLASMIN & FN1 & macular holes & Phase IV & Enzyme \\ 
		DINOPROSTONE & PTGER4 & pregnancy & Phase IV & Small molecule \\ 
		INTERFERON ALFA-2A & IFNAR2 & lymphoma & Phase IV & Protein \\ 
		COPANLISIB & PIK3CG & non-Hodgkins lymphoma & Phase IV & Small molecule \\ 
		INSULIN ZINC SUSP RECOMBINANT HUMAN & INSR & diabetes mellitus & Phase IV & Protein \\ 
		INTERFERON ALFA-N3 & IFNAR2 & severe acute respiratory syndrome & Phase IV & Protein \\ 
	\end{tabularx}
	\smallskip
	\caption{List of drugs targeting the 36 genes. Only Phase IV drugs are reportes.}
	\label{tab:drugs}
\end{table}


\begin{table}[ht]
	\centering
	\scriptsize
	\rowcolors{2}{white}{NavyBlue!10}
	\begin{tabularx}{\textwidth}{XlXl}
		\rowcolor{NavyBlue!80}
		\textbf{\color{white} GO term} & \textbf{\color{white} p-value} & \textbf{\color{white} GO term} & \textbf{\color{white} p-value} \\
		response to organic substance & 1.7e-09 & positive regulation of response to external stimulus & 2.1e-03 \\ 
		cellular response to organic ubstance & 3.0e-09 & movement of cell or subcellular component & 2.4e-03 \\ 
		leukocyte migration & 3.8e-08 & intracellular signal transduction & 2.6e-03 \\ 
		circulatory system development & 5.3e-08 & regulation of transport & 2.6e-03 \\ 
		cellular response to chemical stimulus & 1.4e-07 & cellular response to environmental stimulus & 2.7e-03 \\ 
		regulation of multicellular organismal process & 4.6e-07 & cellular response to abiotic stimulus & 2.7e-03 \\ 
		response to chemical & 7.0e-07 & cell communication & 2.7e-03 \\ 
		blood circulation & 1.8e-06 & regulation of phosphorylation & 2.7e-03 \\ 
		multicellular organismal process & 2.0e-06 & cellular response to external stimulus & 2.8e-03 \\ 
		circulatory system process & 2.2e-06 & regulation of cytokine production & 2.9e-03 \\ 
		myeloid leukocyte migration & 2.5e-06 & cellular response to biotic stimulus & 2.9e-03 \\ 
		cellular response to oxygen-containing compound & 3.2e-06 & interferon-gamma production & 3.0e-03 \\ 
		cellular response to organonitrogen compound & 4.2e-06 & system process & 3.1e-03 \\ 
		response to external stimulus & 4.4e-06 & cellular developmental process & 3.2e-03 \\ 
		cellular response to endogenous stimulus & 8.3e-06 & response to bacterium & 3.4e-03 \\ 
		response to endogenous stimulus & 8.4e-06 & cell proliferation & 3.4e-03 \\ 
		response to organonitrogen compound & 1.1e-05 & cellular response to cytokine stimulus & 3.6e-03 \\ 
		immune response & 1.1e-05 & regulation of cell adhesion & 3.7e-03 \\ 
		cellular response to mechanical stimulus & 1.3e-05 & cellular response to acid chemical & 3.8e-03 \\ 
		protein phosphorylation & 1.6e-05 & leukocyte chemotaxis & 3.8e-03 \\ 
		cellular response to nitrogen compound & 2.3e-05 & collagen-activated signaling pathway & 3.9e-03 \\ 
		cell surface receptor signaling pathway & 3.3e-05 & secretion & 4.2e-03 \\ 
		regulation of response to cytokine stimulus & 3.6e-05 & response to mechanical stimulus & 5.0e-03 \\ 
		response to nitrogen compound & 4.7e-05 & regulation of heart contraction & 5.3e-03 \\ 
		extracellular structure organization & 4.9e-05 & positive regulation of biological process & 5.5e-03 \\ 
		response to stimulus & 5.4e-05 & blood coagulation & 5.6e-03 \\ 
		immune system process & 5.5e-05 & positive regulation of cell migration & 5.6e-03 \\ 
		response to oxygen-containing compound & 6.7e-05 & regulation of secretion by cell & 6.1e-03 \\ 
		multi-organism process & 6.9e-05 & hemostasis & 6.1e-03 \\ 
		regulation of cell communication & 7.0e-05 & positive regulation of nitric-oxide synthase biosynthetic process & 6.2e-03 \\ 
		tube morphogenesis & 7.1e-05 & coagulation & 6.4e-03 \\ 
		regulation of signaling & 8.4e-05 & positive regulation of cell motility & 7.2e-03 \\ 
		wound healing & 8.6e-05 & cell activation & 7.4e-03 \\ 
		response to cytokine & 9.8e-05 & defense response & 7.8e-03 \\ 
		blood vessel development & 1.1e-04 & positive regulation of inflammatory response & 7.9e-03 \\ 
		response to organic cyclic compound & 1.3e-04 & negative regulation of relaxation of cardiac muscle & 8.2e-03 \\ 
		positive regulation of multicellular organismal process & 1.4e-04 & positive regulation of secretion by cell & 8.3e-03 \\ 
		cellular response to stimulus & 1.5e-04 & positive regulation of cellular component movement & 9.0e-03 \\ 
		regulation of immune response & 1.7e-04 & localization & 9.1e-03 \\ 
		anatomical structure development & 1.7e-04 & relaxation of cardiac muscle & 9.3e-03 \\ 
		vasculature development & 1.8e-04 & response to other organism & 9.9e-03 \\ 
		response to lipopolysaccharide & 1.9e-04 & response to external biotic stimulus & 1.0e-02 \\ 
		cardiovascular system development & 2.0e-04 & regulation of phosphate metabolic process & 1.1e-02 \\ 
		granulocyte migration & 2.0e-04 & regulation of phosphorus metabolic process & 1.1e-02 \\ 
		negative regulation of multicellular organismal process & 2.1e-04 & positive regulation of cell adhesion & 1.1e-02 \\ 
		regulation of cellular component movement & 2.3e-04 & regulation of secretion & 1.2e-02 \\ 
		cell migration & 2.4e-04 & regulation of defense response & 1.2e-02 \\ 
		extracellular matrix organization & 2.7e-04 & regulation of system process & 1.2e-02 \\ 
		phosphorylation & 2.7e-04 & peptidyl-tyrosine phosphorylation & 1.3e-02 \\ 
		regulation of response to stimulus & 2.8e-04 & positive regulation of interferon-gamma production & 1.3e-02 \\ 
		response to molecule of bacterial origin & 2.9e-04 & negative regulation of cytokine production & 1.3e-02 \\ 
		regulation of localization & 3.2e-04 & regulation of MAPK cascade & 1.3e-02 \\ 
		regulation of immune system process & 3.2e-04 & nitric-oxide synthase biosynthetic process & 1.3e-02 \\ 
		cardiac muscle contraction & 3.4e-04 & regulation of nitric-oxide synthase biosynthetic process & 1.3e-02 \\ 
		signaling & 3.8e-04 & peptidyl-tyrosine modification & 1.3e-02 \\ 
		signal transduction & 4.2e-04 & positive regulation of protein phosphorylation & 1.3e-02 \\ 
		blood vessel morphogenesis & 4.4e-04 & positive regulation of secretion & 1.4e-02 \\ 
		response to wounding & 4.6e-04 & cytokine-mediated signaling pathway & 1.4e-02 \\ 
		tube development & 5.2e-04 & immune effector process & 1.5e-02 \\ 
		locomotion & 5.5e-04 & positive regulation of immune response & 1.5e-02 \\ 
		heart contraction & 6.0e-04 & response to biotic stimulus & 1.5e-02 \\ 
		regulation of signal transduction & 6.4e-04 & regulation of locomotion & 1.6e-02 \\ 
		cytokine production & 6.5e-04 & cellular response to lipid & 1.6e-02 \\ 
		MAPK cascade & 6.8e-04 & macromolecule modification & 1.8e-02 \\ 
		system development & 7.3e-04 & chemotaxis & 1.8e-02 \\
		heart process & 7.4e-04 & taxis & 1.8e-02 \\	
		localization of cell & 7.5e-04 & regulation of cytokine production involved in immune response & 1.8e-02 \\ 
	\end{tabularx}
	\smallskip
	\caption{GO terms enriched in the list of 36 (continued)}
	\label{tab:GO36_1}
\end{table}
			
\begin{table}[!htb]
	\centering
	\scriptsize
	\rowcolors{2}{white}{NavyBlue!10}
	\begin{tabularx}{\textwidth}{XlXl}
		\rowcolor{NavyBlue!80}
		\textbf{\color{white} GO term} & \textbf{\color{white} p-value} & \textbf{\color{white} GO term} & \textbf{\color{white} p-value} \\
		cell motility & 7.5e-04 & regulation of protein modification process & 1.9e-02 \\ 
		regulation of cytokine-mediated signaling pathway & 7.5e-04 & positive regulation of phosphorylation & 2.0e-02 \\ 
		platelet activation & 7.8e-04 & enzyme linked receptor protein signaling pathway & 2.4e-02 \\ 
		transmembrane receptor protein tyrosine kinase signaling pathway & 8.1e-04 & regulation of response to external stimulus & 2.4e-02 \\ 
		developmental process & 8.2e-04 & positive regulation of intracellular signal transduction & 2.4e-02 \\ 
		signal transduction by protein phosphorylation & 8.3e-04 & positive regulation of response to macrophage colony-stimulating factor & 2.5e-02 \\ 
		response to lipid & 8.4e-04 & positive regulation of cellular response to macrophage colony-stimulating factor stimulus & 2.5e-02 \\ 
		positive regulation of locomotion & 9.3e-04 & positive regulation of signal transduction & 2.6e-02 \\ 
		regulation of blood circulation & 9.6e-04 & positive regulation of immune system process & 2.8e-02 \\ 
		anatomical structure formation involved in morphogenesis & 9.8e-04 & regulation of intracellular signal transduction & 2.9e-02 \\ 
		phosphate-containing compound metabolic process & 1.0e-03 & mononuclear cell migration & 3.3e-02 \\ 
		phosphorus metabolic process & 1.1e-03 & regulation of cardiac muscle contraction & 3.3e-02 \\ 
		multicellular organism development & 1.1e-03 & lymphocyte activation involved in immune response & 3.4e-02 \\ 
		cell chemotaxis & 1.1e-03 & positive regulation of phosphorus metabolic process & 3.5e-02 \\ 
		positive regulation of cell proliferation & 1.1e-03 & positive regulation of phosphate metabolic process & 3.5e-02 \\ 
		cellular response to lipopolysaccharide & 1.1e-03 & regulation of biological quality & 3.5e-02 \\ 
		positive regulation of cytokine production & 1.2e-03 & tissue development & 3.9e-02 \\ 
		regulation of protein phosphorylation & 1.3e-03 & regulation of cell migration & 4.0e-02 \\ 
		anatomical structure morphogenesis & 1.4e-03 & protein modification process & 4.1e-02 \\ 
		angiogenesis & 1.5e-03 & cellular protein modification process & 4.1e-02 \\ 
		positive regulation of response to stimulus & 1.5e-03 & transport & 4.2e-02 \\ 
		cell differentiation & 1.5e-03 & muscle system process & 4.2e-02 \\ 
		cellular response to molecule of bacterial origin & 1.5e-03 & positive regulation of production of molecular mediator of immune response & 4.2e-02 \\ 
		secretion by cell & 1.5e-03 & regulation of leukocyte migration & 4.2e-02 \\ 
		regulation of cell proliferation & 1.6e-03 & positive regulation of metabolic process & 4.6e-02 \\ 
		collagen-activated tyrosine kinase receptor signaling pathway & 1.7e-03 & cytokine production involved in immune response & 4.6e-02 \\ 
		regulation of interferon-gamma production & 1.7e-03 & positive regulation of developmental process & 4.8e-02 \\ 
		striated muscle contraction & 1.9e-03 & positive regulation of immune effector process & 4.8e-02 \\ 
		inflammatory response & 2.0e-03 &  &  \\
	\end{tabularx}
	\smallskip
	\caption{GO terms enriched in the list of 36.}
	\label{tab:GO36_2}
\end{table}

\begin{table}[!htb]
		\centering
		\scriptsize
		\rowcolors{2}{NavyBlue!10}{white}
		\begin{tabularx}{\textwidth}{XlXl}
		\rowcolor{NavyBlue!80}
		\textbf{\color{white} GO term} & \textbf{\color{white} p-value} & \textbf{\color{white} GO term} & \textbf{\color{white} p-value} \\
		cellular response to organic substance & 5.1e-07 & response to interleukin-6 & 9.1e-03 \\ 
		cellular response to chemical stimulus & 5.8e-06 & signaling & 9.2e-03 \\ 
		response to organic substance & 7.6e-06 & cardiac muscle contraction & 9.5e-03 \\ 
		response to organonitrogen compound & 4.9e-05 & regulation of system process & 9.6e-03 \\ 
		regulation of multicellular organismal process & 5.7e-05 & response to external stimulus & 1.0e-02 \\ 
		cellular response to oxygen-containing compound & 1.2e-04 & regulation of transport & 1.1e-02 \\ 
		leukocyte migration & 1.3e-04 & regulation of cardiac muscle cell contraction & 1.2e-02 \\ 
		collagen-activated tyrosine kinase receptor signaling pathway & 1.3e-04 & extracellular matrix organization & 1.4e-02 \\ 
		response to nitrogen compound & 1.4e-04 & response to organic cyclic compound & 1.6e-02 \\ 
		circulatory system development & 2.2e-04 & regulation of actin filament-based movement & 1.6e-02 \\ 
		response to endogenous stimulus & 2.8e-04 & platelet activation & 1.6e-02 \\ 
		collagen-activated signaling pathway & 3.1e-04 & regulation of immune response & 1.9e-02 \\ 
		cellular response to organonitrogen compound & 3.8e-04 & acute-phase response & 2.0e-02 \\ 
		relaxation of cardiac muscle & 7.4e-04 & cellular response to lipopolysaccharide & 2.1e-02 \\ 
		cell surface receptor signaling pathway & 8.4e-04 & negative regulation of relaxation of muscle & 2.1e-02 \\ 
		response to chemical & 9.4e-04 & cellular response to molecule of bacterial origin & 2.5e-02 \\ 
		cellular response to endogenous stimulus & 1.0e-03 & cell migration & 2.6e-02 \\ 
		cellular response to nitrogen compound & 1.2e-03 & striated muscle contraction & 3.0e-02 \\ 
		negative regulation of relaxation of cardiac muscle & 1.4e-03 & extracellular structure organization & 3.0e-02 \\ 
		regulation of cellular component movement & 1.5e-03 & regulation of relaxation of cardiac muscle & 3.0e-02 \\ 
		heart contraction & 3.3e-03 & regulation of interleukin-1-mediated signaling pathway & 3.0e-02 \\ 
		regulation of localization & 3.5e-03 & myeloid leukocyte migration & 3.2e-02 \\ 
		cellular response to stimulus & 3.7e-03 & cellular response to cytokine stimulus & 3.2e-02 \\ 
		response to oxygen-containing compound & 3.8e-03 & signal transduction & 3.6e-02 \\ 
		heart process & 3.8e-03 & movement of cell or subcellular component & 3.6e-02 \\ 
		regulation of blood circulation & 4.7e-03 & cellular response to biotic stimulus & 4.0e-02 \\ 
		blood circulation & 5.4e-03 & system process & 4.2e-02 \\ 
		positive regulation of multicellular organismal process & 5.9e-03 & positive regulation of interferon-gamma production & 4.2e-02 \\ 
		relaxation of muscle & 5.9e-03 & regulation of cell communication & 4.5e-02 \\ 
		circulatory system process & 6.0e-03 & cellular response to acid chemical & 4.8e-02 \\ 
		negative regulation of multicellular organismal process & 6.4e-03 & regulation of signaling & 5.0e-02 \\ 
		cellular response to interleukin-6 & 6.5e-03 &  &  \\ 
	\end{tabularx}
	\smallskip
	\caption{GO terms enriched in the list of 18}
	\label{tab:GO18}
\end{table}


\end{document}